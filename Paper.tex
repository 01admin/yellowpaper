\documentclass[9pt,oneside]{amsart}
%\usepackage{tweaklist}
\usepackage{url}
\usepackage{xspace}
\usepackage{graphicx}
\usepackage{multicol}
\usepackage{subfig}
\usepackage{amsmath}
\usepackage{amssymb}
\usepackage[a4paper,width=170mm,top=18mm,bottom=22mm,includeheadfoot]{geometry}
\usepackage{booktabs}
\usepackage{array}
\usepackage{verbatim}
\usepackage{caption}
\usepackage{natbib}
\usepackage{float}
\usepackage{pdflscape}
\usepackage{mathtools}

\DeclarePairedDelimiter{\ceil}{\lceil}{\rceil}
\newcommand*\eg{e.g.\@\xspace}
\newcommand*\Eg{e.g.\@\xspace}
\newcommand*\ie{i.e.\@\xspace}
%\renewcommand{\itemhook}{\setlength{\topsep}{0pt}  \setlength{\itemsep}{0pt}\setlength{\leftmargin}{15pt}}

\title{Ethereum: A Secure Decentralised Compute Platform}
\author{
    Gavin Wood\\
    CTO, Ethereum Project\\
    gavin@ethereum.org
}
\begin{document}

\begin{abstract}
The block-chain paradigm when coupled with cryptographically-secured transactions has demonstrated its utility through a number of projects, not least Bitcoin. Each such project can be seen as a simple application on a decentralised, but singleton, compute resource. We can call this paradigm a transactional singleton machine with shared-state.

Ethereum implements this paradigm in a generalised manner. Furthermore it provides a plurality of such resources, each with a distinct state and operating code but able to interact through a message-passing framework with others. We discuss its design, implementation issues, the opportunities it provides and the future hurdles we envisage.
\end{abstract}

\maketitle

\setlength{\columnsep}{20pt}
\begin{multicols}{2}

\section{Introduction}\label{sec:introduction}

With ubiquitous internet connections in most places of the world, global information transmission has become incredibly cheap. Technological-rooted movements like Bitcoin have demonstrated through the power of the default, consensus mechanisms and voluntary respect of the social contract that it is possible to use the internet to make a decentralised value-transfer system, shared across the world and virtually free to use. This system can be said to be a very specialised version of a cryptographically secure, transaction-based state machine. Follow-up systems such as namecoin adapted this original ``currency application'' of the technology into other applications allbeit rather simplistic ones.

Ethereum is a project that attempts to build the generalised technology; technology on which all transaction-based state machine concepts may be built. Moreover it aims to provide to the end-developer a tightly integrated end-to-end system for building software on a hitherto unexplored compute paradigm in the mainstream: a trustful object messaging compute framework.

\subsection{Driving Factors} \label{ch:driving}

Transparency. Open state. User can be guaranteed that backend will do what front-end says it will. Parties can analyse and be wholly confident in outcomes given events.

\subsection{Previous Work} \label{ch:previous}

E language. Smart contracts. Bitcoin, namecoin \&c. Mastercoin.

\section{Proposal} \label{ch:proposal}

\subsection{The Block Chain Paradigm} \label{ch:overview}

Ethereum, taken as a whole, can be viewed as a transaction-based state machine: we begin with a genesis state and incrementally execute transactions to morph it into some final state. It is this final state which we accept as the canonical ``version'' of the world of Ethereum. The state can include such information as account balances, reputations, trust arrangements, data pertaining to information of the physical world; in short, anything that can currently be represented by a computer is admissable. Transactions thus represent a valid arc between two states; the `valid' part is important---there exist far more invalid state changes than valid state changes. Invalid state changes might, \eg be things such as reducing an account balance without an equal and opposite increase elsewhere. We define a valid state transitions as one which comes about through a transaction. Formally:

\begin{equation}
S_{t+1} \equiv f(S_t, T)
\end{equation}

and $f$ is the Ethereum state transition function. In Ethereum, $f$, together with $S$ are considerably more powerful then any existing comparable system; $f$ allows components to carry out arbitrary computation to be done, while $S$ allows components to store arbitrary state between transactions.

Transactions are collated into blocks; we then chain blocks together using a cryptographic hash as a means of reference. Blocks function as a journal, recording a series of transactions together with the previous block and an identifier for the final state (though do not store the final state itself---that would be far too big). They also punctuate the transaction series with incentives for nodes to \textit{mine}. This incentivisation takes places as a state-transition function, adding value to a nominated account.

Mining is the process of dedicating effort (working) to bolster one series of transactions (a block) over any other potential competitor block. It is achieved thanks to a cryptographically secure proof. This scheme is known as a proof-of-work and is discussed in detail in section \ref{ch:pow}.

Formally, we expand to:

\begin{eqnarray}
S_{t+1} & \equiv & F(S_t, \mathbb{T}) \\
\mathbb{T} & \equiv & [ T_0, T_1, ... ] \\
F(S, \mathbb{T}) & \equiv & g(f(f(S, T_0), T_1) ...)
\end{eqnarray}

Where $g$ is the block-finalisation state transition function (a function that rewards a nominated party); $\mathbb{T}$ is the series of transactions included in this block; and $F$ is the block-level state-transition function.

This is the basis of the block-chain paradigm, a model that forms the backbone of Ethereum.

\subsubsection{Value}

In order to incentivise computation within the network, there needs to be an agreed method for transmitting value. To address this issue, Ethereum has an intrinsic currency, Ether, known also as ETH. The smallest subdenomination of Ether, and thus the one in which all integer values of the currency are counted is the Wei. One Ether is defined as being $10^{18}$ Wei. There exist other subdenominations of Ether:

\par\bigskip
\begin{center}
\begin{tabular}{lr}
\toprule
Multiplier & Name \\
\midrule
$10^0$ & Wei \\
$10^{12}$ & Szabo \\
$10^{15}$ & Finney \\
$10^{18}$ & Ether \\
\bottomrule
\end{tabular}
\end{center}
\par

Throughout the document, any reference to value, in the context of Ether, currency, a balance or a payment, should assumed to be counted in Wei. 

\subsubsection{Which History?}

Since the system is decentralised and all parties have an opportunity to create a new block on some older pre-existing block the resultant structure is necessarily a tree of blocks. In order to form a consensus as to which path, from root (the genesis block) to leaf (the block containing the most recent transactions) through this tree structure, known as the block-chain, there must be an agreed-upon scheme. If there is ever a disagreement between nodes as to which root-to-leaf path down the block-tree is the `best' block-chain, then a \textit{fork} occurs.

This would mean that past a given point in time (block), there is a temporal co-existance of multiple states of the system: some nodes believing one block to contain the canonical transactions, other nodes believing some other block to be canonical, potentially containing radically different or incompatible transactions. This is to be avoided at all costs as the uncertainty that would ensue would likely kill all confidence in the entire system.

The scheme we use in order to generate consensus is a simplified version of the GHOST algorithm \cite{ghost}. This process is described in detail in section \ref{ch:ghost}.

\subsection{Blocks, State and Transactions} \label{ch:bst}

Having introduced the basic concepts behind Ethereum, we will discuss the meaning of a transaction, a block and the state in more detail.

\subsubsection{World State} \label{ch:state}

The world-state (\textit{state}, see Appendix \ref{app:state}), is a mapping between addresses (160-bit identifiers) and account states (a data structure serialised as RLP). Though not stored on the block chain, it is assumed that the implementation will maintain this mapping in a modified Merkle Patricia tree (\textit{trie}, see Appendix \ref{app:trie}). The trie requires a simple database back-end that maintains a mapping of bytearrays to bytearrays; we name this underlaying database the state database. This has a number of benefits; firstly the root node of this structure is cryptographically dependent on all internal data and as such its hash can be used as a secure identity for the entire system state. Secondly, being an immutable data structure, it allows any previous state (whose root hash is known) to be recalled by simply altering the root hash accordingly. Since we store all such root hashes in the block chain, we are able to trivially revert to old states.

The account state comprises the first two, and potentially the last two, of the following fields:

\begin{description}
\item[nonce] A scalar value equal to the number of transactions sent from this address, or, in the case of contract accounts, the number of contract-creations made by this account.
\item[balance] A scalar value equal to the number of wei owned by this address.
\item[stateRoot] A 256-bit hash equal to the root node of a further trie structure that encodes the storage contents of the contract. Though a separate data structure, this trie is still stored using the same underlying state database. This trie takes the form as a simple mapping between domains of 256-bit values.
\item[codeHash] The hash of the EVM code of this contract---this is the code that gets executed should this address reveive a call; it is immutable and thus, unlike all other fields, cannot be changed after construction. All such code fragments are contained in the state database under their corresponding hashes for later retrieval.
\end{description}

If the latter two fields are missing, the node represents a simple (non-contract) account: it maintains no state and executes no code on receipt of a transaction.

\subsubsection{Transaction} \label{ch:transaction}

A transaction is a single cryptographically signed instruction sent by an actor external to Ethereum. e.g. This might be from a person (via a mobile device or desktop computer) or could be from a piece of automated software running on a server. There are two types of transactions: those which result in message calls and those which result in the creation of new contracts. Both types specify a number of common fields:

\begin{description}
\item[nonce] A scalar value equal to the number of transactions sent by the sender.
\item[value] A scalar value equal to the number of wei to be transferred to the message call's recipient or, in the case of contract creation, as an endowment to the newly created contract's account.
\item[gasPrice] A scalar value equal to the number of wei to be paid per unit of gas for all computation costs incurred as a result of the execution of this transaction.
\item[gasLimit] A scalar value equal to the maximum amount of gas that should be used in executing this transaction. This is paid up-front, before any computation is done and may not be topped-up later.
\item[to] The 160-bit address of the message call's recipient or the zero address for a contract-creation transaction.
\item[v, r, s] Three scalar values corresponding to the signature of the transaction and used to determine the sender of the transaction. See Appendix \ref{app:signing}.
\end{description}

Additionally, a contract-creation transaction contains:

\begin{description}
\item[body] An unlimited size byte array specifying the EVM-code for the contract.
\item[init] An unlimited size byte array specifying the EVM-code for the contract initialisation procedure.
\end{description}

Both \textbf{body} and \textbf{init} are EVM-code fragments; \textbf{body} is executed each time the contract account receives a message call (either through a transaction or due to the internal execution of code). \textbf{init} is a piece of code executed only once and at contract creation. It gets discarded immediately.

Whereas a message call transaction contains:

\begin{description}
\item[data] An unlimited size byte array specifying the input data of the message call.
\end{description}


\subsubsection{Block} \label{ch:block}

The block in Ethereum is the collection of relevant pieces of information (known as the block \textit{header}) together with a set of transactions and a set of other block headers that are known to have a parent equal to the present block's parent's parent (such blocks are known as \textit{uncles}). The block header contains several pieces of information:

\begin{description}
\item[parentHash] The SHA3 256-bit hash of the parent block, in its entirety.
\item[unclesHash] The SHA3 256-bit hash of the uncles list portion of this block.
\item[coinbase] The 160-bit address to which all fees collected from the successful mining of this block be transferred.
\item[stateRoot] The SHA3 256-bit hash of the root node of the state trie, after all transactions are executed and finalisations applied.
\item[transactionsHash] The SHA3 256-bit hash of the transactions list portion of this block.
\item[difficulty] A scalar value corresonding to the difficulty level of this block. This can be calculated from the previous block's difficulty level and the timestamp.
\item[timestamp] A scalar value equal to the reasonable output of Unix's time() at this block's inception.
\item[extraData] An arbitrary byte array containing data relavant to this block. Generally 256-bit or less, but could be much larger in the genesis block.
\item[nonce] A 256-bit hash which proves that a sufficient amount of computation has been carried out on this block.
\end{description}

The difficulty of a block $H$, is defined as $H_d$:

\begin{equation}
H_d :=  \begin{cases}
2^{22} & \quad \text{if genesis block}\\
H'_d + \frac{H'_d}{1024} & \quad \text{if $H_t < H'_t + 42$}\\
H'_d - \frac{H'_d}{1024} & \quad \text{otherwise}\\
\end{cases}
\end{equation}

Where $H'$ is the previous block, $H_t$ is the timestamp of block $H$.

This mechanism enforces a homeostasis in terms of the time between blocks; a smaller period between the last two blocks results in an increase in the difficulty level and thus additional computation required, lengthening the likely next period. Conversely, is the period is too large the difficulty, and expected time to the next block, is reduced.

The nonce, $n$, must satisfy the relation:

\begin{equation}
\mathcal{P}(H, n) < \frac{2^{256}}{H_d}
\end{equation}

Where $\mathcal{P}$ is the proof-of-work function (see section \ref{ch:pow}): this evaluates to an pseudo-random number cryptographically dependent on the parameters $H$ and $n$. Given an approximately uniform distribution in the range $[0, 2^{256})$, then the expected time to find a solution is proportional to the difficulty, $H_d$.

This is the foundation of the security of the blockchain and is the fundamental reason why a malicious node cannot propogate newly created blocks that would otherwise overwrite (``rewrite'') history. Because the nonce must satisfy this requirement, and because satisfaction of this requirement both depends on the contents of the block and 

The other two components in the block are simply a list of the transactions (as defined previously) and a list of uncle headers (of the same format as above).

\subsection{Gas and Payment} \label{ch:payment}

In order to avoid issues of network abuse and to sidestep the inevitable questions stemming from Turing completeness, all programmable computation in Ethereum is subject to fees. The fee schedule is specified in units of \textit{gas} (see Appendix \ref{app:fees} for the fees associated with various computation). Thus any given fragment of programmable computation (this includes creating contracts, making message calls, utilising and accessing contract storage and executing operations on the virtual machine) has a universally agreed cost in terms of gas.

Every transaction has a specific amount of gas associated with it: \textbf{gasLimit}. This is the amount of gas which is implicitly purchased from the sender's account balance. The purchase happens at the according \textbf{gasPrice}, also specified in the transaction. The transaction is considered invalid if the account balance cannot support such a purchase. It is named \textbf{gasLimit} since any unused gas at the end of the transaction is refunded (at the same rate of purchase) to the sender's account. Gas does not exist outside of the execution of a transaction. Thus for trusted contracts, a relatively high gas limit may be set and left alone.

Any ether used to purchase gas that is not refunded is delivered to the \textit{coinbase} address, the address of an account typically under control of the miner. Transactors are free to specify any \textbf{gasPrice} that they wish, however miners are free to ignore transactions as they choose. A higher gas price on a transaction will therefore cost the sender more in terms of Ether and deliver a greater value to the miner and thus will more likely be selected for inclusion by more miners. Miners, in general, will choose to advertise the minimum gas price for which they will execute transactions and transactors will be free to canvas these prices in determining what gas price to offer. Since there will be a (weighted) distribution of minimum acceptable gas prices, transactors will necessarily have a trade-off to make between lowering the gas price and maximising the chance that their transaction will be mined in a timely manner.

%\subsubsection{Determining Computation Costs}

\subsection{Transaction Execution} \label{ch:execution}

The execution of a transaction is the most complex part of the Ethereum protocol: it defines the state transition function $f$ as specified before. It is assumed that any transactions executed first pass the initial tests of intrinsic validity. These include:

\begin{enumerate}
\item The transaction signature is valid;
\item the transaction nonce is valid (equivalent to the sender account's current nonce);
\item the gas limit is no smaller that the intrinsic gas, $i$, used by the transaction;
\item the sender account balance contains at least the cost, $c$, (in wei) required in up-front payment;
\end{enumerate}

where:

\begin{eqnarray}
i & \equiv & \begin{cases}
  |\mathbf{d}|.G_{data} + G_{call} & \quad \text{call} \\
  (|\mathbf{i}| + |\mathbf{b}|).G_{data} + G_{create} & \quad \text{create} \\
\end{cases} \\
c & \equiv & l.p + v
\end{eqnarray}

where $|\mathbf{d}|$, $|\mathbf{i}|$ and $|\mathbf{b}|$ are the sizes, in bytes, of the transaction's associated data, initialisation \& body EVM-code and $G$ is defined in Appendix \ref{app:fees}.

The execution of a valid transaction begins with an irrevocable change made to the state: the sender's nonce is incremented by one and the balance is reduced by the cost, $c$. The gas available for the proceeding computation, $g$, is defined as $l - i$. The proceeding computation, whether contract creation or a message call, results in an eventual state (which may legally be equivalent to the current state), the change to which is deterministic and never invalid: there can be no invalid transactions from this point.

After the message call or contract creation is processed, the state is finalised by refunding $g'$, the remaining gas, to the sender at the original rate, thus we increase the sender's balance by $g.p$.

\subsubsection{Create} \label{ch:create}

There are number of intrinsic parameters used when creating a contract: sender ($s$), nonce ($o$), available gas ($g$), endowment ($v$) together with the two arbitrary length byte arrays, $\mathbf{i}$ for the initialisation EVM code and $\mathbf{b}$ for the EVM code of the body.

The address of the new contract is defined as being the leftmost 160 bits of the SHA3 hash of RLP encoding of the structure \texttt{[ sender, nonce ]}. In in the unlikely event that the addresses is already in use, it is treated as a big-endian integer and incremented by one until an unused address is arrived at, thus:

\begin{eqnarray}
a' & \equiv & \mathcal{L}_{160}(\mathcal{S}(\mathcal{R}([s, o])))\\
a & \equiv & \arg \min_x : x \geq a' \wedge \mathbb{S}[x] = \emptyset
\end{eqnarray}

where $\mathcal{S}$ is the SHA3 256-bit hash function, $\mathcal{R}$ is the RLP encoding function, $\mathcal{L}_y(X)$ evaluates to the leftmost $y$ bits of the binary data $X$ and $\mathbb{S}[x]$ is the address state of $x$ or $\emptyset$ if none exists.

The account's nonce is initially defined as zero, the balance as the value passed, the storage as empty and the code hash as the SHA3 256-bit hash of the code, thus the new state becomes $\mathbb{S}'$:

\begin{equation}
\mathbb{S}'[r] := ( 0, v, \empty, \mathcal{S}(\mathbf{b}) )
\end{equation}

The state database is altered to include the pair $(\mathcal{S}(\mathbf{b}), \mathbf{b})$. Finally, the contract is initialised through the execution of the initialising EVM code $\mathbf{i}$ according to the execution model (see section \ref{ch:model}). Code execution can effect several events that are not internal to the execution state: the contract's storage can be altered, further contracts can be created and further message calls can be made.

Code execution depletes gas; thus it may exit before the code has come to a natural halting state. In this exceptional case we say an Out-of-Gas exception has occured and the state database is reverted to the point immediately prior to the account initialisation. All available gas is used and no gas is refunded to the originator. If such an exception does not occur, then the remaining gas is refunded to the originator and the now-altered state is allowed to persevere.

\subsubsection{Message Call} \label{ch:call}

In the case of executing a message call, several parameters are required: sender ($s$), recipient ($r$), available gas ($g$), value ($v$) together with an arbitrary length byte array, $\mathbf{d}$, the input data of the call. Message calls also have intrinsic return data; the call's output byte array $\mathbf{r}$. This is ignored when executing transactions, however message calls can be initiated due to VM-code execution and in this case the return data is used.

In executing a message call, the account balance of the recipient is increased by the value passed, thus $\mathbb{S}$ becomes replaced with $\mathbb{S}'$:

\begin{equation}
\mathbb{S}'[r]_{balance} := \mathbb{S}[r]_{balance} + v
\end{equation}

In the case that the recipient contains code to be executed (i.e. that the account is a contract), then the contract's body code, to be found in the state database with key $\mathbb{S}[r]_{code}$, is executed according to the execution model (see section \ref{ch:model}). Just as with contract creation, if the execution halts due to an exhausted gas supply, then no gas is refunded to the caller and the state is reverted to the point immediately prior to code execution. Unlike with contract creation, 

\subsection{Execution Model} \label{ch:model}

The execution model specifies how the system state is altered given a series of bytecode instructions and a small tuple of environmental data. This is specified through a formal model of a virtual state machine, known as the Ethereum Virtual Machine (EVM). It is a \textit{quasi-}Turing-complete machine; the \textit{quasi} qualification comes from the fact that the computation is intrinsically bounded through a parameter, \textit{gas}, which limits the total amount of computation done.

\subsubsection{Basics}

The EVM is a simple stack-based architecture. The word size of the machine (and thus size of stack item) is 256-bit. This was chosen to facilitate the SHA3-256 hash scheme. The memory model is a simple word-addressed byte array. The stack has an unlimited size. The machine also has an independent storage model; this is similar in concept to the memory but rather than a byte array, it is a word-addressable word array. Unlike memory, which is volatile, storage is non volatile and is maintained as part of the system state in the state database. All locations in both storage and memory are well-defined initially as zero.

The machine does not follow the standard von Neumann architecture. Rather than storing program code in memory or storage, it is stored separately in a virtual ROM. It cannot be read directly; instead it exists only as a model for determining the next instruction to execute.

The machine can have exceptional execution for several reasons, including stack underflows, invalid instructions and divides by zero. These unambiguously and validly result in immediate halting of the machine with all state changes left intact. The one piece of exceptional execution that does not leave state changes intact is the out-of-gas (OOG) exception. Here, the machine halts immediately and reports the issue to the execution agent (either the transaction processor or, recursively, the spawning execution environment) and which will deal with it separately.

\subsubsection{Fees Overview}

Fees (denominated in gas) are charged under three distinct circumstances, all three as prerequisite to the execution of an operation. The first and most common is the fee intrinsic to the computation of the operation. Most operations require a single gas fee to be paid for their execution; exceptions include {\small SSTORE}, {\small SLOAD}, {\small CALL}, {\small CREATE}, {\small BALANCE} and {\small SHA3}. Secondly, gas may be deducted in order to form the payment for a subordinate message call or contract creation; this forms part of the payment for {\small CREATE} and {\small CALL}. Finally, gas may be paid due to an increase in the usage of the memory.

Over a contract's execution, the total fee for memory-usage payable is proportional to smallest multiple of 32 bytes that are required such that all memory indices (whether for read or write) are included in the range. This is paid for on a just-in-time basis; as such, referencing an area of memory at least 32 bytes greater than any previously indexed memory will certainly result is an addition memory usage fee. Due to this fee it's highly unlikely addresses will ever go above 32-bit bounds since at the present price of ether and default gas price, that would cost around US\$20M for the memory fee alone.

Storage fees have a slightly nuanced behaviour---to incentivise contracts to minimise storage use (which corresponds directly to a larger state database on all nodes), the execution fee for an operation that clears an entry in the storage is waived; in fact, it is effectively paid up-front since the initial usage of a storage location costs twice as much as the normal usage.

More formally, given an instruction, it is possible to calculate the gas cost of executing it as follows:

\begin{itemize}
\item {\small SHA3} costs $G_{sha3}$ gas
\item {\small SLOAD} costs $G_{sload}$ gas
\item {\small BALANCE} costs $G_{balance}$ gas
\item {\small SSTORE} costs $d.G_{sstore}$ gas where:
\begin{itemize}
\item $d = 2$ if the new value of the storage is non-zero and the old is zero;
\item $d = 0$ if the new value of the storage is zero and the old is non-zero;
\item $d = 1$ otherwise.
\end{itemize}
\item {\small CALL} costs $G_{call} + g$, where $g$ is the quantity of gas provided; some gas may later be refunded.
\item {\small CREATE} costs $G_{create} + g$ gas, where $g$ is the quantity of gas provided.
\item All other operations cost $G_{step}$ gas.
\end{itemize}

Additionally, when memory is accessed with {\small MSTORE}, {\small MLOAD}, {\small RETURN}, {\small SHA3}, {\small CALLDATA} or {\small CALL}, the memory should be enlarged to the smallest multiple of words such that all addressed bytes now fit in it. The memory size must always be a whole number of 32-byte words. Suppose that the highest previously accessed memory index is $i$, and the new index is $i'$; an additional $G_{m}(i, i')$ gas must be paid:

\begin{equation}
G_{m}(i, i') \equiv \max(0, \ceil[\Bigg]{\frac{i'+1}{32}} - \ceil[\Bigg]{\frac{i+1}{32}}) G_{memory}
\end{equation}

Originally, with no accesses taken place, the highest accessed index of memory $i$ is defined as $-1$.

%Whenever a higher memory index is referenced, the fee difference to take it to the higher usage from the original (lower) usage is charged. Notably, because {\small MSTORE} and {\small MLOAD} operate on word lengths, they implicitly increase the highest-accessed index to 31 greater than their target index.

\subsubsection{Execution Environment}

In addition to the system state $\mathbb{S}$, there are several pieces of important, immutable, information used in the execution environment that the execution agent must provide:

\begin{itemize}
\item $a$, the address of the account which own the code that is executing.
\item $o$, the sender address of the transaction that originated this execution.
\item $p$, the price of gas in the transaction that originated this execution.
\item $D$, the byte array that is the input data to this execution; if the execution agent is a transaction, this would be the transaction data.
\item $s$, the address of the account which caused the code to be executing; if the execution agent is a transaction, this would be the transaction sender.
\item $v$, the value passed to this account as part of the same procedure as execution; if the execution agent is a transaction, this would be the transaction value.
\item $H$, the block header of the previous block.
\item $H'$, the block header of the current block.
\item $n$, the block number of the current block.
\end{itemize}

The execution model defines the function $X$, which can compute the resultant state $\mathbb{S}'$, given the required information:

\begin{equation}
\mathbb{S}' \equiv X(\mathbb{S}, a, o, p, D, s, v, H, H', n)
\end{equation}

\subsubsection{Execution Overview}



\begin{enumerate}
\item Let $\mathbb{S}' := \mathbb{S}$, the current state.
\item Let $PC := 0$, the program counter.
Let it be known that all storage operations operate on TO's state.
Note that the memory is empty, thus (last used index + 1) = 0.
\item Set TXDATA to the first INSIZE bytes of memory starting from INOFFSET in the caller memory.
\item Repeat
\begin{itemize}
\item Calculate the cost of the current instruction (see below), set to $C$.
\item If the instruction is invalid or STOP, goto step 6.
\item If $GAS < C$ then $GAS := 0$; $S := ROLLBACK$ and evaluation finishes, returning false. 
\item $GAS := GAS - C$
\item Apply the instruction.
\end{itemize}
Until an operation execution error has occured or the instruction is STOP, RETURN or SUICIDE.
\item If the output data of the call (R) is specified (through RETURN), then let the OUTSIZE bytes of caller memory beginning at OUTOFFSET.
\item Returns true.
\end{enumerate}



When a contract address receives a transaction, a virtual machine is initiated with the contract's state.

\subsubsection{Terms}

There exists a stack of variable size that stores 256-bit values at each location. (Note: most instructions operate solely on the stack altering its state in some way.)

T'[i] is the ith item counting down from the top of the pre-stack (i.e. the stack immediately after instruction execution), with the top of the stack corresponding to i == 0.

T[i] is the ith item counting down from the top of the post-stack (i.e. the stack immediately prior to instruction execution), with the top of the stack corresponding to i == 0.


The exists a permanent store addressable by a 256-bit value that stores 256-bit values at each location.

S'[i] is the permanent store (sometimes refered to as 'state' or 'store') of the VM at index i counting from zero PRIOR to instruction execution.

S[i] is the permanent store (sometimes refered to as 'state' or 'store') of the VM at index i counting from zero AFTER to instruction execution.


The exists a temporary store addressable by a 256-bit value that stores 256-bit values at each location.

M'[i] is the temporary store (sometimes refered to as 'memory') of the VM at index i counting from zero PRIOR to instruction execution.

M[i] is the temporary store (sometimes refered to as 'memory') of the VM at index i counting from zero AFTER instruction execution.

C is the code, a byte sequence.

PC' is the program counter PRIOR to instruction execution.

PC is the program counter AFTER instruction execution.


FEE(I, S', P', D) is the fee associated with the execution of instruction I with a machine of stack S', permanent store P' and which has already completed D operations.

It is defined as F where:

TODO: No it's not!

IF I == SSTORE AND P[ S'[0] ] != 0 AND S'[1] == 0 THEN
    F = S + dataFee - memoryFee
IF I == SSTORE AND P[ S'[0] ] == 0 AND S'[1] != 0 THEN
    F = S + dataFee + memoryFee
IF I == SLOAD
    F = S + dataFee
IF I == EXTRO OR I == BALANCE
    F = S + extroFee
IF I == MKTX
    F = S + txFee
IF I == SHA256 OR I == SHA3 OR I == RIPEMD160 OR I == ECMUL OR I == ECADD
    OR I == ECSIGN OR I == ECRECOVER OR I == ECVALID THEN
    F = S + cryptoFee 

Where:

B[ A ] is the balance of the address given by A, with A interpreted as an address.

ADDRESS is the address of the present contract.


10. b. Initial Operation

STEPSDONE := 0
PC' := 0
FOR ALL i: M' is initialised to the empty byte sequence.
T' is initialised such that its cardinality is zero (i.e. the stack starts empty).
S' must be equal to the value of P when the previous execution halted.


10. c. General Operation

The given steps are looped:
1. Execution halts if B[ ADDRESS ] < F( C[PC'], S', P', STEPSDONE )
2. B[ ADDRESS ] := B[ ADDRESS ] - F( P'[PC'], S', P', STEPSDONE )
3. The operation given by P'[PC'] determines PC, P, T, S.
4. PC' := PC; P' := P; T' := T; S' := S; STEPSDONE := STEPSDONE + 1


10. d. VM Operations

Summary line:
<Op-code>: <Mnemonic> -<R> +<A>

If PC is not defined explicitly, then it must be assumed PC := PC' + 1. Exceptions are PUSH, JUMP and JUMPI.

The cardinality of T (i.e. size of the stack) is altered by A - R between T \& T', by adding or removing items as necessary from the front.

Where:
R: The minimal cardinality of the stack for this instruction to proceed. If this is not achieved then the machine halts with a stack underflow exception. (Note: In some cases of some implementations, this is also the number of values "popped" from the implementation's stack during the course of instruction execution.)
(A - R): The net change in cardinality of the stack over the course of instruction execution.

$\forall$ i: if T[i] is not defined explicitly, then it must be assumed T[i] := T'[i + R - A] where i + R >= A.
$\forall$ i: if M[i] is not defined explicitly, then it must be assumed M[i] := M'[i].
$\forall$ i: if S[i] is not defined explicitly, then it must be assumed S[i] := S'[i].

The expression (COND ? ONE : ZERO), where COND is an expression and ONE and ZERO are both value placeholders, evaluates to ONE if COND is true, and ZERO if COND is false. This is similar to the C-style ternary operator.

When a 32-byte machine datum is interpreted as a 160-bit address or hash, the rightwards 20 bytes are taken (i.e. the low-order bytes when interpreting the data as Big-Endian).

++ is the concatenation operator; all operands are byte arrays (mostly 32-byte arrays here, since that's the size of the VM's data \& address types).

LEFT\_BYTES(A, n) returns the array bytes comprising the first (leftmost) n bytes of the 32 byte array A, which can be considered equivalent to a single value in the VM.



\subsection{Block-tree to Block-chain} \label{ch:ghost}

The canonical block chain is a path from root to leaf through the entire block tree. In order to have consensus over which path it is, conceptually we identify the path that has had the most computation done upon it, or, the \textit{heaviest} path. Clearly one factor that helps determine the heaviest path is the block number of the leaf, equivalent to the number of blocks, not counting the unmined genesis block, in the path. The longer the path, the greater the total mining effort that must have been done in order to arrive at the leaf. This is akin to existing schemes, such as that employed in Bitcoin-derived protocols.

This scheme notably ignores so-called \textit{stale} blocks: valid, mined blocks, which were propogated too late into the network and thus were beaten to network consensus by a sibling block (one with the same parent). Such blocks become more common as the network propogation time approaches the ideal inter-block time. However, by counting the computation work of stale block headers, we are able to do better: we can utilise this otherwise wasted computation and put it to use in helping to butress the more popular block-chain making it a stronger choice over less popular (though potentially longer) competitors.

This increases overall network security by making it much harder for an adversary to silently mine a canonical block chain (which, it is assumed, would contain different transactions to the current consensus) and dump it on the network with the effect of reversing existing blocks and the transactions within.

In order to validate the extra computation, a given block $B$ may include the block headers from any known uncle blocks (i.e. blocks whose parent is equivalent to the grandparent of $B$). Since a block header includes the nonce, a proof-of-work, then the header alone is enough to validate the computation done. Any such blocks contribute toward the total computation or \textit{total difficulty} of a chain that includes them. To incentivise computation and inclusion, a reward is given both to the miner of the stale block and the miner of the block that references it.

Thus we define the total difficulty of block $B$ recursively as:

\begin{eqnarray}
B_{td} & \equiv & P_{td} + B_d + \sum\limits_{U \in B_\mathbb{U}} U_d \\
P & \equiv & B_P
\end{eqnarray}

As such given a block $B$, $B_td$ is its total difficulty, $B_P$ is its parent block, $B_d$ is its difficulty and $B_\mathbb{U}$ is its set of uncle blocks.

\subsection{Block Finalisation} \label{ch:finalisation}

Validate uncles.

Reward everything.

Compute proof-of-work.

\subsection{Mining Proof-of-Work} \label{ch:pow}

The mining proof-of-work (PoW) exists as a cryptographically secure nonce that proves beyond reasonable doubt that a particular amount of computation has been expended in the determination of some token value $n$. It is utilised to enforce the block chain security by giving meaning and credence to the notion of difficulty (and, by extension, total difficulty). However, since mining new blocks comes with an attached reward, the proof-of-work not only functions as a method of securing confidence that the blockchain will remain canonical into the future, but also as a wealth distribution mechanism.

For both reasons, there are two important goals of the proof-of-work function; firstly, it should be as accessible as possible to as many people as possible. The requirement of or reward from specialised and uncommon hardware should be minimised. This makes the distribution model as open as possible, and, ideally, makes the act of mining a simple swap from electricity to ether at roughly the same rate for anyone around the world.

Secondly, it should not be possible to make super-linear profits, and especially not so with a high initial barrier. Such a mechanism allows a well-funded adversary to gain a troublesome amount of the network's total mining power and as such gives them a super-linear reward (thus skewing distribution in their favour) as well as reducing the network security.

One plague of the Bitcoin world is ASICs. These are specialised pieces of compute hardware that exist only to do a single task. In Bitcoin's case the task is the SHA256 hash function. While ASICs exist for a proof-of-work function, both goals are placed in jeopardy. Because of this, a proof-of-work function that is ASIC-resistant (i.e. difficult or economically inefficient to implement in specialised compute hardware) has been identified as the proverbial silver bullet.

Two directions exist for ASIC resistance; firstly make it sequential memory-hard, i.e. engineer the function such that the determination of the nonce requires a lot of memory and that the memory cannot be used in parallel to discover multiple nonces simultaneously. The second is to make the type of computation it would need to do general-purpose; the meaning of ``specialised hardware''  for a general-purose task set is, naturally, general purpose hardware and as such commodity desktop computers are likely to be pretty close to ``specialised hardware'' for the task. 

More formally, the proof-of-work function takes the form of $\mathcal{P}$:

\begin{equation}
\mathcal{P}(H, n) < \frac{2^{256}}{H_d}
\end{equation}

Where $H$ is the new block's header \textit{without} the nonce entry; $H_d$ is the difficulty value (i.e. the block difficulty from section \ref{ch:ghost}).

As of the proof-of-concept (PoC) series of the Ethereum software, the proof-of-work function is simplistic and does not attempt to secure these goals. It will be described here for completeness.

\subsubsection{PoC Series}

For the PoC series, we use a simplified proof-of-work. This is not ASIC resistant and is meant merely as a placeholder. It utilises the SHA3 hash function to 

It is formally defined as $\mathcal{P}$:

\begin{equation}
\mathcal{P}(H, n) \equiv \mathcal{B}(\mathcal{S}(\mathcal{S}(\mathcal{R}(H)) \bigoplus n))
\end{equation}

where:
$\mathcal{R}(H)$ is the RLP encoding of the $H$;
$\mathcal{S}$ is the SHA3 hash function accepting an arbitrary length series of bytes and evaluating to a series of 32 bytes (i.e. 256-bit);
$n$ is the nonce, a series of 32 bytes;
$\bigoplus$ is the series concatenation operator;
$\mathcal{B}(X)$ evaluates to the value equal to X when interpreted as a big-endian-encoded integer.

\subsubsection{Release Series}

For the release series, we use a more complex proof-of-work. This has yet to be formally defined, but involves two components; firstly, 

\section{User Interface}

Browser-style.

Displays ETH balance (i.e. functions as a wallet).

Tabs or perhaps MacOS style app icons.

Display taken up with a single app.

Extensible \eg should be able to display the balances of other currencies hosted on Etheruem alongside ETH.

Should be able to trust certain apps with creation of transactions (and thus sending of ETH \& payment of gas).

Preconfigured with home app that is the Eth app store.

Eth app store comes preloaded with high trust for ethereum.org; apps have identity and WoT reputation.

\section{Implementing Contracts}

\subsection{Data Feeds}

External server runs node. Creates \& signs a transaction every minute containing new data. Sends to contract which knows to accept data only from this address. Allows polling of data from all other contracts (possibly for a microfee).

\subsection{Random Numbers}

Hash the block timestamp; for a series of pseudorandoms, take the previous, add some constant and hash the result.

For a more secure pseudo-random offering both parties agree on a number of random data feed contracts; these are concatenated along with the block timestamp and hashed to produce the first number in the series.

\section{Discussion} \label{ch:discussion}

\subsection{Examples} \label{ch:examples}

\section{Conclusion} \label{ch:conclusion}

\subsection{Further Work} \label{ch:further}

\bibliography{Biblio}
\bibliographystyle{plainnat}

\appendix

\section{Terminology}

\begin{description}
\item[External Actor] A person or other entity able to interface to an Ethereum node, but external to the world of Ethereum. It can interact with Ethereum through depositing signed Transactions and inspecting the block-chain and associated state. Has one (or more) intrinsic Accounts.

\item[Address] A 160-bit code used for identifying Accounts.

\item[Account] Accounts have an intrinsic balance and transaction count maintained as part of the Ethereum state. They are owned either by External Actors or intrinsically (as an identity) an Autonomous Object within Ethereum. If an Account identifies an Autonomous Object, then Ethereum will also maintain a Storage State particular to that Account. Each Account has a single Address that identifies it.

\item[Transaction] A piece of data, signed by an External Actor. It represents either a Message or a new Autonomous Object. Transactions are recorded into each block of the block-chain.

\item[Autonomous Object] A virtual object existant only within the hypothetical state of Ethereum.  Has an intrinsic address. Incorporated only as the state of the storage component of the VM.

\item[Storage State] The information particular to a given Autonomous Object that is maintained between the times that it runs.

\item[Message] Data (as a set of bytes) and Value (specified as Ether) that is passed between two Accounts in a perfectly trusted way, either through the deterministic operation of an Autonomous Object or the cryptographically secure signature of the Transaction.

\item[Message Call] The act of passing a message from one Account to another. If the destination account is an Autonomous Object, then the VM will be started with the state of said Object and the Message acted upon. If the message sender is an Autonomous Object, then the Call passes any data returned from the VM operation.

\item[Gas] The fundamental network cost unit. Paid for exclusively by Ether (as of PoC-4), which is converted freely to and from Gas as required. Gas does not exist outside of the internal Ethereum computation engine; its price is set by the Transaction and miners are free to ignore Transactions whose Gas price is too low.

\item[Contract] Synonym for Autonomous Object used for non-technical audiences.

\item[Object] Synonym for Autonomous Object.

\item[App] An end-user-visible application hosted in the Ethereum Browser.

\item[Ethereum Browser] (aka Ethereum Reference Client) A cross-platform GUI of an interface similar to a simplified browser (ala Chrome) that is able to host sandboxed applications whose back-end is purely on the Ethereum protocol.

\item[Ethereum Virtual Machine] (aka EVM) The virtual machine that forms the key part of the execution model for a contract's program code.

\item[EVM Code] The bytecode that the EVM can natively execute.

\item[EVM Assembly] The human-readable form of EVM-code.

\item[LLL] The Lisp-like Low-level Language, a human-writable language used for authoring simple contracts.

\end{description}

\section{Recursive Length Prefix}\label{app:rlp}

\section{Modified Merkle Patricia Tree}\label{app:trie}

\section{Hex-Prefix Encoding}\label{app:hexprefix}

\section{Formal Specification of Structures}

Though RLP is data-agnostic, it does specify a canonical representation for integer quantities. It is big-endian with no leading zero-bytes. Thus for elements than feasibly can be stored as integers, it becomes important to specify whether they should adhere to this canonical representation or be left in some other (perhaps more 'native') format.

In the case of counts, balances, fees and amounts of wei, the canon-integer form must be used when storing in RLP. We call these INTs.

In the case of hashes (256-bit or 160-bit), user-visible strings and specialised byte-arrays (e.g. hex-prefix notation from the trie), they should be stored as unformatted byte-array data and not altered into some other form. We call these BINs.

When interpreting RLP data, clients are required to consider non-canonical INT fields in the same way as otherwise invalid RLP data and dismiss it completely.

Specifically:

for the Block header:
\begin{verbatim}
[
  parentHash: BIN (256-bit),
  unclesHash: BIN (256-bit),
  coinbase: BIN (160-bit),
  stateRoot: BIN (256-bit),
  transactionsHash: BIN (256-bit),
  difficulty: INT,
  timestamp: INT,
  extraData: BIN (typically 256-bit),
  nonce: BIN (256-bit)
]
\end{verbatim}

(note: 'nonce', the last element, refers to a hash here and so is binary)

for entries in the State trie for normal addresses:
\begin{verbatim}
[
  balance: INT,
  nonce: INT
]
\end{verbatim}

and for contract addresses:
\begin{verbatim}
[
  balance: INT,
  nonce: INT,
  storageRoot: BIN (256-bit),
  codeHash: BIN (256-bit)
]
\end{verbatim}

(note: 'nonce', the second element, refers to a tx-count here and so is integer)

for message call transactions:

\begin{verbatim}
[
  nonce: INT,
  value: INT,
  gasPrice: INT,
  gas: INT,
  recvAddr: BIN (160-bit),
  data: BIN (arbitrary length),
  v: INT,
  r: INT,
  s: INT
]
\end{verbatim}

or, for contract creation transactions,

\begin{verbatim}
[
  nonce: INT,
  value: INT,
  gasPrice: INT,
  gas: INT,
  0: BIN (160-bit),
  code: BIN (arbitrary length),
  init: BIN (arbitrary length),
  v: INT,
  r: INT,
  s: INT
]
\end{verbatim}

(note: 'nonce', the first element, refers to a tx-count here and so is integer)

The nonce in the transaction refers to the total amount of transactions sent from the address up until that moment in time.

for blocks, there are no immediate data field, but lists:

\begin{verbatim}
[
  blockHeader: [...]
  uncleList: [ [...], [...], ... ]
  txList: [ [...], [...], ... ]
]
\end{verbatim}

Uncle-blocks contain only the uncle's header.

\section{The State DB}\label{app:state}

\section{Signing Transactions}\label{app:signing}

\section{Virtual Machine Specification}\label{app:vm}

0s: arithmetic operations

0x00: STOP -0 +0
Halts execution.
Any gas left over gets returned to caller (or in the case of the top-level call, the sender converted back to ETH).
0x01: ADD -2 +1
S[0] := S'[0] + S'[1]
0x02: MUL -2 +1
S[0] := S'[0] * S'[1]
0x03: SUB -2 +1
S[0] := S'[0] - S'[1]
0x04: DIV -2 +1
S[0] := S'[0] / S'[1]
0x05: SDIV -2 +1
S[0] := S'[0] / S'[1]
S'[0] \& S'[1] are interpreted as signed 256-bit values for the purposes of this operation.
0x06: MOD -2 +1
S[0] := S'[0] % S'[1]
0x07: SMOD -2 +1
S[0] := S'[0] % S'[1]
S'[0] \& S'[1] are interpreted as signed 256-bit values for the purposes of this operation.
0x08: EXP -2 +1
S[0] := S'[0] + S'[1]
0x09: NEG -1 +1
S[0] := -S'[0]
0x0a: LT -2 +1
S[0] := S'[0] < S'[1] ? 1 : 0
0x0b: GT -2 +1
S[0] := S'[0] > S'[1] ? 1 : 0
0x0c: EQ -2 +1
S[0] := S'[0] == S'[1] ? 1 : 0
0x0d: NOT -1 +1
S[0] := S'[0] == 0 ? 1 : 0

10s: bit operations

0x10: AND -2 +1
S[0] := S'[0] AND S'[1]
0x11: OR -2 +1
S[0] := S'[0] OR S'[1]
0x12: XOR -2 +1
S[0] := S'[0] XOR S'[1]
0x13: BYTE -2 +1
S[0] := S'[0]th byte of S'[1]
if S'[0] < 32
S[0] := 0
otherwise
for Xth byte, we count from left - 0th is therefore the leftmost (most significant in BE) byte.

20s: crypto opcodes

0x20: SHA3 -2 +1
S[0] := SHA3( T'[ S'[0] ++ ... ++ (S'[0] + S'[1]) ])

30s: closure state opcodes

0x30: ADDRESS -0 +1
S[0] := ADDRESS
i.e. the address of this closure.
0x31: BALANCE -1 +1
S[0] := B[ S'[0] ]
i.e. the balance of this closure.
0x32: ORIGIN -0 +1
S[0] := A
Where A is the address of the account that made the original transaction leading to the current closure and is paying the fees
0x33: CALLER -0 +1
S[0] := A
Where A is the address of the object that made this call.
0x34: CALLVALUE -0 +1
S[0] := V
Where V is the value attached to this call.
0x35: CALLDATALOAD -1 +0
S[0] := D[ S'[0] ... (S'[0] + 31) ]
Where D is the data attached to this call (as a byte array).
Any bytes that are out of bounds of the data are defined as zero.
0x36: CALLDATASIZE -0 +1
S[0] := DS
Where DS is the number of bytes of data attached to this call.
0x37: GASPRICE -0 +1
S[0] := V
Where V is the current gas price (né fee multiplier).

40s: block operations

0x40: PREVHASH -0 +1
S[0] := H
Where H is the SHA3 hash of the previous block.
0x41: COINBASE -0 +1
S[0] := A
Where A is the coinbase address of the current block.
0x42: TIMESTAMP -0 +1
S[0] := T
Where T is the timestamp of the current block (given as the Unix time\_t when this block began its existence).
0x43: NUMBER -0 +1
S[0] := N
Where N is the block number of the current block (counting upwards from genesis block which has N == 0).
0x44: DIFFICULTY -0 +1
S[0] := D
Where D is the difficulty of the current block.
0x45: GASLIMIT -0 +1
S[0] := L
Where L is the total gas limit of the current block. Always $10^6$.

50s: stack, memory, storage and execution path operations

0x51: POP -1 +0
0x52: DUP -1 +2
S[0] := S'[0]
0x53: SWAP -2 +2
S[0] := S'[1]
S[1] := S'[0]
0x54: MLOAD -1 +1
S[0] := T'[ S'[0] ... S'[0] + 31 ]
0x55: MSTORE -2 +0
T[ S'[0] ... S'[0] + 31 ] := S'[1]
0x56: MSTORE8 -2 +0
T[ S'[0] ... S'[0] + 31 ] := S'[1] \& 0xff
0x57: SLOAD -1 +1
S[0] := P'[ S'[0] ]
0x58: SSTORE -2 +0
P[ S'[0] ] := S'[1]
0x59: JUMP -1 +0
PC := S'[0]
0x5a: JUMPI -2 +0
PC := S'[1] == 0 ? PC' : S'[0]
0x:5b PC -0 +1
S[0] := PC
0x5c: MSIZE -0 +1
S[0] = sizeof(T)
0x5d: GAS
S[0] := G
Where G is the amount of gas remaining after executing the opcode.

60s \& 70s: push

0x60: PUSH1 0 +1
S[0] := C[ PC' + 1 ]
PC := PC' + 2
0x61: PUSH2 0 +1
S[0] := C[ PC' + 1 ] ++ C[ PC' + 2 ]
PC := PC' + 3
...
0x7f: PUSH32 0 +1
S[0] := C[ PC' + 1 ] ++ C[ PC' + 2 ] ++ ... ++ C[ PC' + 32 ]
PC := PC' + 33

f0s: closure-level operations

0xf0: CREATE -5 +1
Immediately creates a contract where:
The endowment is given by S'[0]
The body code of the eventual closure is given by T'[ S'[1] ... ( S'[1] + S'[2] - 1 ) ]
The initialisation code of the eventual closure is given by T'[ S'[3] ... ( S'[3] + S'[4] - 1 ) ]
(Thus the total number of bytes of the transaction data is given by S'[2] + S'[4].)
S[0] = A
where A is the address of the created contract or 0 if the creation failed.
Fees are deducted from the sender balance to pay for enough gas to complete the operation (i.e. contract creation fee + initial storage fee). If there was not enough gas to complete the operation, then all gas will be deducted and the operation fails.
0xf1: CALL -7 +1
Immediately executes a call where:
The recipient is given by S'[0], when interpreted as an address.
The value is given by S'[1]
The gas is given by S'[2]
The input data of the call is given by T'[ S'[3] ... ( S'[3] + S'[4] - 1 ) ]
(Thus the number of bytes of the transaction is given by S'[4].)
The output data of the call is given by T[ S'[5] ... ( S'[5] + MIN( S'[6], S[0] ) - 1 ) ]
If 0 gas is specified, transfer all gas from the caller to callee gas balance, otherwise transfer only the amount given. If there isn't enough gas in the caller balance, operation fails.
If the value is less than the amount in the caller's balance then nothing is executed and the S'[2] gas gets refunded to the caller.
See Appendix A for execution.
Add any remaining callee gas to the caller's gas.
S[0] = R
where R = 1 when the instrinsic return code of the call is true, R = 0 otherwise.
NOT YET: POST -5 +1
Registers for delayed execution a call where:
The recipient is given by S'[0], when interpreted as an address.
The value is given by S'[1]
The gas to supply the transaction with is given by S'[2] (paid for from the current gas balance)
The input data of the call is given by T'[ S'[3] ... ( S'[3] + S'[4] - 1 ) ]
(Thus the number of bytes of the transaction is given by S'[4].)
Contract pays for itself to run at a defered time from its own GAS supply. The miner has no choice but to execute.
NOT YET: ALARM -6 +1
Registers for delayed execution a call where:
The recipient is given by S'[0], when interpreted as an address.
The value is given by S'[1]
The gas (to convert from ETH at the later time) is given by S'[2]
The number of blocks to wait before executing is S'[3]
The input data of the call is given by T'[ S'[4] ... ( S'[4] + S'[5] - 1 ) ]
(Thus the number of bytes of the transaction is given by S'[5].)
Total gas used now is S'[3] * S'[5] * deferFee.
Contract pays for itself to run at the defered time converting given amount of gas from its ETH balance; if it cannot pay it terminates as a bad transaction. TODO: include baseFee and allow miner freedom to determine whether to execute or not. If not, the next miner will have the chance.
0xf2: RETURN -2 +0
Halts execution.
R := T[ S'[0] ... ( S'[0] + S'[1] - 1 ) ]
Where the output data of the call is specified as R.
Any gas left over gets returned to caller (or in the case of the top-level call, the sender converted back to ETH).
0xff: SUICIDE -1 +0
Halts execution.
FOR ALL i: IF P[i] NOT EQUAL TO 0 THEN B[ S'[0] ] := B[ S'[0] ] + memoryFee
B[ S'[0] ] := B[ S'[0] ] + B[ ADDRESS ]
Removes all contract-related information from the Ethereum system.

\section{Low-level Lisp-like Language}\label{app:lll}

\section{Wire Protocol}\label{app:wire}

\section{Peer Strategy}\label{app:peers}

\section{Genesis Block}\label{app:genesis}

The header of the genesis block is 9 items, and is specified thus:

$[z_{256}, \mathcal{S}(\mathcal{R}([])), z_{160}, stateRoot, \mathcal{S}(\mathcal{R}([])), 2^{22}, 0, \text{`'}, 42]$

Where:

$z_{256}$ refers to the parent hash, a 256-bit hash which is all zeroes.

$z_{160}$ refers to the coinbase address, a 160-bit hash which is all zeroes.

$2^{22}$ refers to the difficulty.

0 refers to the timestamp (the Unix epoch).

`' refers to the extradata, an empty byte array.

$\mathcal{S}(\mathcal{R}([]))$ values refer to the hashes of the transaction and uncle lists in RLP, both empty.

$stateRoot$ refers to the state root.

\section{Fee Schedule}\label{app:fees}

Constants:

* STEPGAS = 1

* SHA3GAS = 20

* SLOADGAS = 20

* SSTOREGAS = 100

* BALANCEGAS = 20

* CREATEGAS = 100

* CALLGAS = 20

* MEMORYGAS = 1

* TXDATAGAS = 5 [not used in the VM]

\end{multicols}

\end{document}

